%%%%%%%%%%%%%%%%%%%%%%%%%%%%%%%%%%%%%%%%%%%%%%%%%%%%%%%%%%%%%%%%%%%%%%
%% On jet mass corrections and transversity
%% May 10, 2013
%%%%%%%%%%%%%%%%%%%%%%%%%%%%%%%%%%%%%%%%%%%%%%%%%%%%%%%%%%%%%%%%%%%%%%
\documentclass[preprintnumbers,floatfix,nofootinbib]{revtex4}

\usepackage[tbtags]{amsmath}  % AMS math 
\usepackage{amssymb}          % AMS symbols
\usepackage{bm}               % bold math
\usepackage{graphicx}         % PostScript figures
\usepackage[export]{adjustbox}% to align images
\usepackage{hhline,multirow}  % for nicer tables
\usepackage{dcolumn}          % Align table columns on decimal point
\usepackage{slashed}
\usepackage{datetime}
\usepackage{color}
\usepackage[dvipsnames]{xcolor}

%%%%%% for draft
\newcommand{\todo}[1]{\marginpar{$\bullet$}\textbf{#1}}
\def\AAcom#1{{\bf  \textcolor{Red}{[AA: {#1}]}}}
\def\AAmod#1{{\textcolor{Green}{#1}}}

%%%%%%  Definitions   %%%%%%%%%%%
  
\newcommand{\Pslash}{P \hspace{-0.24cm} / \,}
\newcommand{\kslash}{k \hspace{-0.21cm} / \,}
\newcommand{\lslash}{l \hspace{-0.19cm} / \,}
\newcommand{\nslash}{n \hspace{-0.24cm} / \,}
\newcommand{\vslash}{v \hspace{-0.21cm} / \,}
\newcommand{\Sslash}{S \hspace{-0.22cm} / \,}
\newcommand{\Dslash}{D \hspace{-0.22cm} / \,}
\newcommand{\epsslash}{\varepsilon \hspace{-0.18cm} / \,}
\newcommand{\Tr}{\operatorname*{Tr}\nolimits} % Trace operator
\newcommand{\xbj}{x}                   % Bjorken variable
\newcommand{\de}{d}                    % differential
\newcommand{\ii}{i}                    % imaginary unit

\newcommand{\eg}{{\em e.g.}}
%\newcommand{\mj}{\langle m_j \rangle}
%\newcommand{\mj}{M_q^\text{jet}}
\newcommand{\mj}{M_q}
\newcommand{\mq}{m_q}
\newcommand{\mjs}{\langle m_j^2 \rangle>}

\allowdisplaybreaks[2]

\begin{document}

%%%%%%%%%%%%%%%%%%%%%%%%%%%%%%%%%%%%%%%%%%%%%%%%%%%%%%%%%%%%%%%%%%%%%%
%%%%%%%%%%%%%%%%%%%%%%%   TITLE    %%%%%%%%%%%%%%%%%%%%%%%%%%%%%%%%%%%
%%%%%%%%%%%%%%%%%%%%%%%%%%%%%%%%%%%%%%%%%%%%%%%%%%%%%%%%%%%%%%%%%%%%%%

\preprint{JLAB-THY-16-????}

\title{Accessing the nucleon tensor structure in inclusive deep inelastic scattering} 

\author{Alberto~Accardi$^{a}$, Alessandro~Bacchetta$^{b}$} 
\affiliation{
$^a$Hampton University, Hampton, VA 23668, USA,
and Jefferson Lab, Newport News, VA 23606, USA \\
$^b$Dipartimento di Fisica, Universit\`a degli Studi di Pavia, and INFN,
Sez. di Pavia, 27100 Pavia, Italy
}

\date{\bf \today, \currenttime}

\begin{abstract}
We revisit the standard analysis of inclusive DIS on protons taking into account the fact that on-shell quarks cannot be present in the final state, but they rather decay into hadrons. As a consequence, a spin-flip term associated with the invariant mass of this (mini)jet of hadrons is generated non perturbatively, and couples to the target's transversity distribution function. In inclusive cross sections, this provides an hitherto neglected and large contribution to the twist-3 part of the $g_2$ structure function, that can explain the discrepancy between recent calculations and fits of this quantity. It also provides an extension of the Burkardt-Cottingham sum rule, that puts stringent constraints on the small-$x$ behavior of the transversity function. Perspectives to measure the new spin flip term, and applications to spin-1 targets will be briefly discussed.
\end{abstract}

%\pacs{ }

%\keywords{ }

\maketitle


%%%%%%%%%%%%%%%%%%%%%%%%%%%%%%%%%%%%%%%%%%%%%%%%%%%%%%%%%%%%%%%%%%%%%%%%%
\section{Introduction}

The tensor charge is a fundamental property of the nucleon, at
present poorly constrained. It has been estimated in lattice
QCD (see, \eg,
\cite{Green:2012ej,Bali:2014nma,Bhattacharya:2015wna,Abdel-Rehim:2015owa,Bhattacharya:2016zcn}), but
only limited information is available from direct measurements. The
way to extract the tensor charge from experimental measurements requires first
of all the extraction the so-called transversity parton distribution function,
denoted by $h_1^q(x)$ (see Ref.~\cite{Barone:2001sp} for a review on transversity and
Refs.~\cite{Radici:2015mwa,Anselmino:2015sxa,Kang:2015msa} for the most recent
extractions). 
The integral of the transversity distribution
corresponds to the contribution of flavor $q$ to the tensor charge.
The knowledge of the tensor charge can be used also to put constraints on the search of physics beyond the Standard
Model~\cite{Cirigliano:2013xha,Bhattacharya:2015esa,Courtoy:2015haa}.

In this paper, we discuss the possibility of observing the effect of
the transversity parton distribution function (PDF) in totally inclusive Deep Inelastic Scattering.

The transversity distribution is notoriously difficult to access because it is
a chiral-odd function and needs to be combined with a spin-flip mechanism to
appear in a scattering process~\cite{Jaffe:1996zw}. Usually, this spin flip is provided by another
nonperturbative distribution or fragmentation function, implying that
transversity cannot be accessed in inclusive DIS,
but only in more complex processes such as semi-inclusive DIS or Drell-Yan~\cite{Ralston:1979ys,Jaffe:1991kp,Jaffe:1993xb,Collins:1993kk}. 

The only other way to attain spin-flip terms in QED and QCD is taking into
account mass corrections. In fact, it is well known that transversity gives a
contribution to the structure function $g_2$ in inclusive DIS (see, \eg,
\cite{Accardi:2009au} and references therein), and in
particular to the violation of the so-called Wandzura--Wilczek relation for
$g_2$~\cite{Wandzura:1977qf}. However, this contribution is proportional to the current quark mass
and can be expected to be very small.

We revisit the standard analysis of inclusive DIS taking into account the fact
that on-shell quarks cannot be present in the final state, but they rather
decay into hadrons (ideally, forming jets of hadrons). This is sufficient to modify the structure of the DIS
cut-diagram, even if none of the hadrons is detected in the final
state. For a proper description of this effect, we include proper ``jet
correlators'' into the analysis, and pay particular attention to ensuring that our results are gauge invariant. We observe that the inclusion of jet correlators introduces a new
contribution to the inclusive $g_2$ structure function. This term has the
interesting features that: a) violates the Wandzura--Wilczek relation, b)
potentially violates the Burkhardt--Cottingham sum rule, c) is
proportional to the transversity distribution function multiplied by a
nonperturbative ``jet mass'' parameter, likely much larger than the mass of light quarks. We provide estimates of this contribution based on a recent extraction of transversity and show that it could be very large.   
 

%%%%%%%%%%%%%%%%%%%%%%%%%%%%%%%%%%%%%%%%%%%%%%%%%%%%%%%%%%%%%%%%%%%%%%%%%
\section{Jet correlator and twist-2 structure functions}

Motivated by large-$x$ mass corrections to inclusive DIS structure functions,
Accardi and Qiu have introduced in the LO handbag diagram 
a ``jet correlator'' (also called ``jet factor'' by Collins and Rogers in Ref.~\cite{Collins:2007ph})
that accounts for invariant mass production in the current jet, and ensures
that leading twist calculations in collinear factorization are consistent with
the requirement imposed by baryon number conservation that $x_B<1$
\cite{Accardi:2008ne}. The jet correlator is depicted in
Figure~\ref{fig:handbags}(a) and is defined as 
\begin{align} 
\Xi_{ij}(l,n_+) =\int
  \frac{\de^4\eta}{(2\pi)^4}\; e^{i l \cdot \eta}\,
    \langle 0|\, {\cal U}^{n_+}_{(+\infty,\eta)}
\,\psi_i(\eta)
             \bar{\psi}_j(0)\,
{\cal U}^{n_+}_{(0,+\infty)}\,   |0 \rangle \ ,
\label{e:xifull}
\end{align} 
In this definition, $l$ is the quark four-momentum, $\Psi$ the quark field
operator (with quark flavor index omitted for simplicity), and $|0\rangle$ is
the nonperturbative vacuum state. Furthermore, we explicitly guarantee the
correlator's gauge invariance by introducing two Wilson line operators ${\cal
  U}^{n_+}$ along a light-cone plus direction determined by the vector
$n_+$. This path choice for the Wilson line is required by QCD factorization
theorems, and the vector is determined by the particular hard process to which
the jet correlator contributes. For example, in the case of inclusive DIS
discussed in this paper, this is determined by the four momentum transfer $q$
and the proton's momentum $p$. 

The correlator $\Xi$ can be parametrized in terms of jet parton correlation
functions (PCFs), using the vectors $l$ and $n_+$: 
\begin{equation}
\Xi(l,n_+) = \Lambda A_1(l^2)\,{\bm 1} + A_2(l^2)\,\lslash 
+ \frac{\Lambda^2}{l \cdot n_{+}} \nslash_+ \, B_{1}(l^2)
+ \frac{i \Lambda}{2 l \cdot n_{+}} [\,\lslash,\nslash_+ ] \, B_{2}(l^2) \ .
\label{e:jetexpansion}
\end{equation} 
Time reversal invariance in QCD requires $B_{2}=0$, while $B_{1}$ contributes
only at twist-4 order, and will not be considered further in this paper. We
focus, instead, on the role of chiral odd terms in the $g_2$ structure function up to twist 3. At this order, 
\begin{equation}
  \Xi(l,n_+) = \Lambda A_1(l^2)\,{\bm 1} + A_2(l^2)\,\lslash 
    + O(\Lambda^2/Q^2)
\label{e:jetexpansion-tw3}
\end{equation} 
is nothing else than the full quark propagator; note however, that we consider
here the full QCD vacuum rather than the perturbative one.  
The $A_1$ and $A_2$ terms can be nicely interpreted in terms of the spectral
representation of the cut quark propagator (see, e.g., Sec.~6.3 of
\cite{D'Hoker:2004aa} and Sec.~2.7.2 of \cite{Romao:2013aa}),
\begin{align} 
  \Xi(l) =  
  \int d \mu^2 \big[ J_1 (\mu^2)\,\mu + J_2 (\mu^2)\,\lslash \big] \,
  \delta(l^2 -\mu_j^2) \ ,
\label{e:jetspectral}
\end{align}
where $\mu^2$ is interpreted as the invariant mass of the current jet, {\it
  i.e.}, of the particles going through the cut in the top blob of
Fig.\ref{fig:handbags}(a), and the $J_i$ are the spectral functions of the
quark propagator, that have been also called ``jet functions'' in
\cite{Accardi:2008ne}. These satisfy
\begin{align}
  J_2(\mu^2) \geq J_1(\mu^2) \geq 0
  \hspace*{0.5cm} \text{and} \hspace*{0.5cm}
  \int d\mu^2 J_2(\mu^2) = 1 \ .
\label{eq:jetfnsprops}
\end{align}
From a comparison of Eqns.\eqref{e:jetexpansion} and \eqref{e:jetspectral}, one can see that 
\begin{align}
  A_1(l^2)&=\frac{\sqrt{l^2}}{\Lambda}J_1(l^2) & A_2(l^2)&=J_2(l^2) \ .
  \label{eq:jet_vs_spectral}
\end{align}

\begin{figure*}[bt]
  \centering
  (a)\includegraphics[width=0.3\linewidth,valign=t]{jetdiagram0}
  \hfill
  (b)\includegraphics[width=0.3\linewidth,valign=t]{jetdiagram2}
  \hfill
  (c)\includegraphics[width=0.3\linewidth,valign=t]{jetdiagram1}
  \caption{Diagrams contributing to DIS scattering up to twist-3 expansion, including a jet correlator in the top part. Note the gluon attaches to both the nucleon and jet correlators. The Hermitian conjugates of diagrams (b) and (c), i.e., with gluons attaching to the right of the cut, are not shown.
  }
  \label{fig:handbags}
\end{figure*}

When inserting the jet correlator in the handbag diagram for inclusive DIS,
the invariant jet mass $\mu^2$ is integrated from 0 to $Q^2(1/x_B-1)$. This
induces (kinematical) corrections of order $O(1/Q^2)$, whose effect on the
$F_2$ structure function has been studied in Ref.~\cite{Accardi:2008ne}: 
\begin{align}
  F_2(x_B) & = \int_0^{Q^2(1/x_B-1)}d\mu^2\, J_2(\mu^2) F_2^{(0)}(x_B(1+\mu^2/Q^2)) \ ,
\label{eq:F2}
\end{align}
where $F_2^{(0)}$ is the structure function calculated with the handbag
diagram sporting a bare quark propagator instead of the jet correlator, and
$\xi=2x_B/(1+\sqrt{1+4x_B^2M^2/Q^2})$ with $M$ the nucleon's mass is the
Nachtmann scaling variable. (We also omitted the dependence of the structure
function on $Q^2$ for clarity of notation). In this paper we limit our
attention to effects of order $O(1/Q)$ and therefore can extend the
integration to $\mu^2=\infty$. Therefore, the jet function $J_2$ decouples
and, thanks to the sum rule \eqref{eq:jetfnsprops}, integrates to 1. One then
recovers the conventional result, 
\begin{align}
  F_2(x_B) = \Big[ \int_0^\infty d\mu^2\, J_2(\mu^2) \Big] F_2^{(0)}(x_B) 
     + O(\Lambda^2/Q^2) = F_2^{(0)}(x_B)  + O(\Lambda^2/Q^2) \ .
\end{align}
%The same holds true also for the helicity structure function $g_1$.
%
More in general, the jet correlator decouples from the parton correlator $\Phi$ in any inclusive cross section calculation up to $O(1/Q)$, and the inclusive structure functions only depend on the integrated jet correlator 
\begin{equation} 
  \Xi(l^-) \equiv \int \frac{dl^2}{2l^-} d^2 l_T \, \Xi(l) 
    =  \frac{\Lambda}{2 l^-}\,\xi_1 {\bm 1}
    +  \xi_2 \frac{\nslash_-}{2} 
    + \text{higher\ twists}
    % +\frac{\Lambda^2}{4 (l^-)^2}\,\xi_3 \nslash_+
    % + i \frac{\Lambda}{2 l^-} \xi_4 
    % \frac{ \bigl[\nslash_-, \nslash_+ \bigr]}{2}.
\end{equation} 
where 
\begin{align}
\xi_1 &= \int d\mu^2 \frac{\mu}{\Lambda} J_1(\mu^2) 
       \equiv \frac{\mj}{\Lambda},
&
\xi_2 &= \int d\mu^2 J_2(\mu^2) = 1 \ .
% \\
% \xi_3 &= 0,
% &
% \xi_4 &= \int d\mu^2 \frac{\mu^2}{\Lambda^2} J_1(\mu^2) = \frac{\mjs}{\Lambda^2}.
\end{align} 
where $\mj$ can be interpreted as the average invariant mass produced in the spin-flip fragmentation processes of a quark of flavor $q$.
% The $\xi_1$ and $\xi_4$ have an interpretation as average invariant mass (and mass squared) of the current jet, and $l^-=q^- + O(1/Q)$ is imposed by 4-momentum conservation at the hard scattering vertex. The $\xi^4$ term contributes at order $O(1/Q^2)$ and is no further considered in this paper. 
It is important to notice that $\xi_2=1$ exactly due to CPT invariance
(see Sec.~10.7 of Ref.~\cite{Weinberg:1995mt}), while $0 < \mj < \int d\mu^2 \mu J_2(\mu^2)$ is
dynamically determined. From the analytic properties of spectral functions we
may expect \cite{Accardi:2008ne} $J_2(\mu^2) = Z \delta(\mu^2-m_q) + \bar J_2
(\mu^2) \theta (\mu^2-m_\pi^2)$ with the continuum starting at $m_\pi$, the
mass of the pion, due to color confinement effects. Taking into account that
$J_1 < J_2$, we may therefore expect  
\begin{align}
  \label{eq:mjet}
  \mj = O(10^2 \text{ MeV}) \ .
\end{align}
Although $\mj$ is in general a nonperturbative quantity, it is interesting to
notice that  
\begin{align}
  \label{eq:xi2_chiral_cond}
  \mj = \frac{\Lambda}{4} \int \Tr \big[ \Xi(l) \bm 1] 
   = \langle 0 | \bar \psi_i(0) \psi_i(0) | 0 \rangle
\end{align}
% \todo{[AA] 2 issues here: (1) possible factors of 4. 
% (2) I am neglecting the sum over quarks; we have to fix that somehow but there is a little tension between the quark level $\xi^a$ and the contribution to structure function which is summed and weighted by the electron charge.} 
Calculating this on the perturbative vacuum and to leading order
corresponds to taking the trace of the cut bare-quark propagator to obtain 
$\mj = \,_{\text{pert}} \langle 0
|  \bar \psi_i(0) \psi_i(0) | 0 \rangle_{\text{pert}}  = \mq$, with $\mq$ the quark mass, recovering the
conventional result. However, we are here considering non perturbative effects
on the quark fragmentation and $\mj \gtrsim \mq$. 
%very importantly, this
%quantity can be computed in lattice QCD. 
%\todo{[AA] Are qe sure that $\mj > \mq$??}
%\todo{[AA] Need to discuss connection to quark condensate and tensor charge.}

\section{Twist-3 analysis}

Extending the analysis of \cite{Accardi:2008ne} to the calculation of twist-3
structure functions requires not only to consider the $\xi_1$ term in the jet
correlator, but also quark-gluon-quark correlators in both the proton and the
vacuum as depicted in Figs.\ref{fig:handbags}(b) and (c), respectively. 
In the former the $\xi_1$ terms contribute to $O(1/Q^2)$, so that up $O(1/Q)$
these give the same contribution as in the conventional handbag calculation.  

The novel element in our analysis is the jet's quark-gluon-quark correlator
$\Xi_A^{\mu}(l,k)$ in diagrams \ref{fig:handbags}(c), 
\begin{equation} 
\begin{split} 
  \left(\Xi_A^{\mu} \right)_{ij} &=
   \frac{1}{2}\, \sum_X \int \frac{\de \eta^+\, \de^2 \bm{\eta}_T}{(2\pi)^{3}}\;
   e^{\ii k \cdot \eta}\,
   \langle 0|\,
   {\cal U}^{n_+}_{(+\infty,\eta)}\,
   g A^{\mu}(\eta)\,
   \,\psi_i(\eta)|X\rangle
   \langle X|
             \bar{\psi}_j(0)\,
   {\cal U}^{n_+}_{(0,+\infty)}
   |0\rangle \bigg|_{\eta^-=0} .
\label{e:xi_A}
\end{split} 
\end{equation}  
This diagram and its Hermitian conjugate are not only important to account for
all contribution of order $O(1/Q)$, but also to restore the
gauge invariance, which is broken in diagram \ref{fig:handbags}(a) due to the different mass
of the incoming and outgoing quark lines, namely, $\mq \neq \mj$. 

Rather than directly using the definition \eqref{e:xi_A}, it is convenient to
calculate the inclusive cross section as an integral of the semi-inclusive
one, utilize the QCD equation of motions and furthermore summed over all
hadron flavors, and take advantage of  
\begin{align}
  \label{eq:SIDIS_to_DIS}
  \sum_h \int \frac{d^3p_h}{(2\pi)2E_h} \Delta^h(l,p_h) = \Xi(l) \ , 
\end{align}
where $\Delta^h$ is the quark fragmentation correlator for production of a
hadron of flavor $h$ and momentum $p_h$ \cite{Bacchetta:2006tn}. In terms of
the TMD fragmentation functions we are interested in, this reads 
\begin{align}
  \label{eq:SIDIS_to_DIS_TMDlevel}
  \sum_h \int dz d^2p_{hT} z D_1^h(z,p_{hT}) & = \xi_2 = 1   \\
  \sum_h \int dz d^2p_{hT} E(z,p_{hT}) & = \xi_1 \ ,
\end{align}
where $D_1^h(z,p_{hT})$ is the twist-2 quark fragmentation function as a
function of the hadron's collinear momentum fraction $z$ and transverse
momentum $p_{hT}$, and $\tilde E^h(z,p_{hT})$ is a chiral-odd twist-3 function
defined in \cite{Bacchetta:2006tn}. 

% As we shall see, an analogous formula for quark-gluon-quark correlators is not needed.
%The $\xi_1$ term is chiral-odd and therefore can appear in the inclusive cross section only coupled to the transversity function $h_1$. Therefore, for our analysis 
The relevant part of the semi-inclusive hadronic tensor for our analysis is 
\begin{align}
  \label{eq:Wsidis_ini}
  2 \Lambda  W^{\mu\nu}
    & = i \frac{2\Lambda}{Q} \hat t^{[\mu}_{\phantom \perp} 
    \epsilon_\perp^{\nu]\rho}S_{\perp\rho} 
    \sum_q e_q^2
    \bigg[ 2 x_b g_T(x_B) \sum_h \int dz d^2p_{hT} D_1^{q,h}(z,p_{hT}) 
  %\\ &
  + 2 h_1(x_B) \sum_h \int dz d^2p_{hT} \tilde E^{q,h}(z,p_{hT}) \bigg] + \ldots
\end{align}
For clarity, here we reintroduced the quark flavor $q$, $e_q$ being its electric charge.
The first term can be easily integrated with the help of the sum rule
\eqref{eq:SIDIS_to_DIS_TMDlevel}. To integrate the latter, we first need make
use of the relation $\tilde E(z) = E(z) - (\mq/\Lambda) z D_1(z)$, which is a
consequence of the QCD equations of motion \cite{Bacchetta:2006tn}, then
make use of the sum rule \eqref{eq:SIDIS_to_DIS}: 
\begin{align}
  \sum_h \int dz d^2p_{hT} \tilde E^{q,h}(z,p_{hT}) 
    = \sum_h \int dz d^2p_{hT} \Big[ E^{q,h}(z,p_{hT}) - \frac{\mq}{\Lambda} z D_1^{q,h}(z,p_{hT}) \Big]
    = \xi_1 - \frac{\mq}{\Lambda} \xi_2 = \frac{\mj - \mq}{\Lambda} \ .
\end{align}
This formula is the single most important result of this paper, and provides a
non perturbative generalization of the commonly used $\int\tilde E =0$ sum
rule introduced in \cite{Jaffe:1996zw}. Indeed, calculating the jet correlator 
on the perturbative vacuum one would obtain, as already discussed, $\mj=\mq$
and the new term would vanish.

Finally, the contraction of the hadronic tensor with the leptonic tensor leads
to the following well known result for the inclusive DIS cross section up to order $\Lambda/Q$~\cite{Bacchetta:2006tn}
\begin{align}
\frac{d\sigma}{d\xbj \, dy\, d\psi}
%&
=
\frac{2 \alpha^2}{\xbj y Q^2}\,
\frac{y^2}{2\,(1-\varepsilon)}\, 
\biggl\{
&F_{UU ,T} + \varepsilon F_{UU ,L}
+ S_\parallel \lambda_e\,
  \sqrt{1-\varepsilon^2}\; 
F_{LL}
%\nonumber 
%\\  &
+ |\bm{S}_\perp| \lambda_e\, \sqrt{2\,\varepsilon (1-\varepsilon)}\, 
  \cos\phi_S\, 
F_{LT}^{\cos \phi_S}
 \biggr\} \ ,
\label{e:crossdis}
\end{align}
where the structure functions on the right hand side correspond to
\begin{align}
F_{UU ,T} &= \xbj\,\sum_q e_q^2\,f_1^q(\xbj),
\\
F_{UU ,L} &= 0,
\\
F_{LL} &=\xbj\,\sum_q e_q^2\,g_1^q(\xbj),
\\
F_{UT}^{\sin \phi_S}&=0,
\label{e:FUTint}
\\
F_{LT}^{\cos \phi_S}&=-\xbj\,\sum_q e_q^2\, \frac{2\Lambda}{Q}\,
\biggl(\xbj  g_T^q(\xbj)
   + \frac{\mj -\mq}{\Lambda} \, h_{1}^q(\xbj) \biggr).
\label{e:FLTint}
\end{align}
The second term in the last structure function is a new result from our
analysis; it is not suppressed as an inverse power of $Q$ compared to the
standard term. On the nonperturbative
vacuum the jet mass is larger than the quark's, and this contributes a
nonnegligible term to the twist-3 part of the $g_2$ function, as we will
discuss in the next section.  

 

\section{The $g_2$ structure function}

The new term in Eq.\eqref{e:FLTint} only appears in the $g_2$ structure function. Following the derivation in Ref.~\cite{Accardi:2009au}, one finds
\begin{align}
\label{e:g2}
  g_2(x_B) = g_2^{WW} + \frac{1}{2}\,\sum_a e_a^2
\biggl(
    \widetilde g_T^{a \star}(x) 
    + \int_x^1\frac{dy}{y} \widehat{g}_T^q(y) 
    + \frac{\mq}{\Lambda} \left(\frac{h_1^q}{x}\right)^\star(x) 
    + \frac{\mj-\mq}{\Lambda} \frac{h_1^q(x)}{x} 
\Biggr) \ ,
\end{align}
where we defined $f^*(x) = -f(x) + \int_x^1\frac{dy}{y} f(y)$. The first four
terms coincide with the result obtained in the conventional handbag
approximation \cite{Accardi:2009au}, while the fifth is new. Note that even if
the relation is written for the sum of the quark weighted by their charge
squared, it can be considered valid also flavor by flavor. In fact, the steps
leading to such a decomposition are formulated at the correlator level.

The first term is also known as the Wandzura-Wilczeck function $g_2^{WW} =
-g_1^*(x)$ , and contains all the twist-2 chiral-even contributions to
the $g_2$ structure coming from quark-quark correlators. The second and third
terms contain all ``pure twist-3'' contributions, i.e., those coming from
quark-gluon-quark correlators. The fourth and fifth terms depend on the
transversity parton distribution function, $h_1$. 
The fourth term is usually neglected for
light quarks since it is proportional to $\mq=O$(1 MeV). In the last term,
new in our analysis, the transversity distribution is multiplied by a constant
of $O$(100 MeV), and cannot be a priori neglected.

It is important to estimate the size of the various contributions to the non Wandzura-Wilczek part of $g_2$. We define the shorthand notation
\begin{align}
g_2^{\rm tw3} & = \frac{1}{2}\,\sum_q e_q^2
    \biggl(
    \widetilde g_T^{q \star}(x) 
    + \int_x^1\frac{dy}{y} \widehat{g}_T^q(y) 
    \biggr) 
&
g_2^{\text{quark}} &= \frac{1}{2}\,\sum_q e_q^2 
 \frac{\mq}{\Lambda} (h_1^q/x)^\star(x),
&
g_2^{\text{jet}} &= \frac{1}{2}\,\sum_q e_q^2 
\frac{\mj-\mq}{\Lambda} \frac{h_1^q(x)}{x}. 
\end{align} 
These are compared in Figure~\ref{f:g2contrib} to the $g_2-g_2^{WW}$ function
obtained in the very recent JAM15 fit of polarized DIS asymmetries
\cite{Sato:2016tuz}, that includes a large amount of precise data at large $x$
from Jefferson Lab, and simultaneously fits the higher-twist components of in
$g_1$ and $g_2$ to the data. For the ``pure twist-3'' contribution,
$g_2^{\rm tw3}$, {\it i.e.}, the contribution from quark-gluon-quark matrix
elements, we show a model calculation by Braun et al.~\cite{Braun:2011aw}; for
(modified) bag model calculations, see \cite{Jaffe:1990qh,Stratmann:1993aw}. To estimate the contributions
from quark ($g_2^q$) and jet mass ($g_2^{\rm jet}$) effects, that depend on chiral
odd quark-quark matrix elements, we use the recent Pavia15 fit of the
transversity distribution from Ref.~\cite{Radici:2015mwa}, which is comparable
also to other 
extractions~\cite{Anselmino:2013vqa,Kang:2015msa}. Furthermore, we choose the
values of the mass parameters to be $\mq=5$ MeV and $\mj = 100$ MeV. 

As one can see, in the proton case the pure twist-3 contribution is quite
smaller in magnitude, and opposite in sign, compared to the JAM15 fit. As
expected, the quark-mass contribution is essentially negligible. 
For what concerns the jet-mass contribution, the uncertainties due to the $h_1$
extraction are very large, especially at low $x$. In addition, there is an
overall normalization uncertainty due to the choice of $\mj$, not shown in the
plot. In any case, the jet-mass contribution is strikingly large. If we assume
that the pure twist-3 contributions are of the order of the model calculation
by Braun et al., the breaking of the Wandura-Wilczek relation can be used to
constrain the extractions of the transversity distribution, in particular at
low $x$. 
Moreover, it is quite clear that the gap
between the pure twist-3 $g_2^{\rm tw3}$ function and the JAM15 fit can be
explained by the new jet-mass contribution we discussed in this paper.  

In the neutron case, the jet contribution is very negative at intermediate to
large values of $x$. 
If one trusts the order of magnitude of the $g_2^{\rm tw3}$
calculation by Braun et al., one would conclude that the jet contribution
should not be that large. However, the jet contribution is strongly influenced
by the $d$ quark's
transversity, whose fit suffers from large systematic uncertainties and
saturates the negative Soffer bound. Recent deata in $p+p$ collisions indicate,
however, that $h_1^d$ might be less negative than in the Pavia15
fits~\cite{Radici:2016lam}.
Correspondingly the jet contribution to the proton at $x \approx 0.1$
would become less positive, inproving as well the agreement with the JAM15
fit. 

\begin{figure}[tbh]
\begin{center}
\includegraphics[width=8cm]{g2contrib}
\includegraphics[width=8cm]{g2contribN}
\caption{\label{f:g2contrib} 
Different contributions to the non Wandzura-Wilczek part of the proton (left)
and neutron (right) $g_2$ structure function compared to the JAM15 fit of the
$g_2-g_2^{\text{WW}}$ (solid black) \cite{Sato:2016tuz}. The quark and jet
contributions are shown with a dotted red and a dot-dashed green line
respectively, with uncertainty bands coming form the Pavia15 fit of the
transversity function \cite{Radici:2015mwa}. The unceratainty in the choice
$m_q=5$ GeV and $M_q=100$ GeV is not shown. The pure twist-3 contribution
calculated by Braun et al. \cite{Braun:2011aw} is shown as a dashed blue line
(no uncertainty estimate was provided in the original reference). 
}
\end{center}
\end{figure}

It is interesting to consider the moments of the non Wandzura-Wilczek contribution to $g_2$,
\begin{align}
  d_N \equiv (N+1) \int_0^1 x^N \bigg( g_2(x) - g_2^{WW}(x) \bigg) \ .
\end{align}
For a generic function $f$, let us define it's $N$-th moment as $f[N]=\int_0^1 dx\, x^{N} f(x)$. It is then straightforward to verify that $f^*[N] = N/(N+1) \times f[N]$ and  
\begin{align}
  d_N & = (N+1) g_2[N] + N g_1[N] \\
  & = \frac12 \sum_q e_q^2 \bigg( N \tilde g_T^q[N] + \hat g_T^q[N]
    + \frac{(N+1) M_q-m_q}{\Lambda} h_1^q[N] \bigg) \ .
\end{align}

The zero-th moment provides an interesting relationship between transversity
and the inclusive structure function $g_2$:
\begin{align}
  \label{eq:BC}
  \int dx\, g_2(x) = \sum_q e_q^2 \frac{\mj-\mq}{\Lambda} \int dx\, \frac{1}{x} h_1^q(x) \ ,
\end{align}
where we used the fact that $\hat g_T^q[0]=0$ identically due to the symmetry
properties of the quark-gluon-quark correlators. 
The sum rule \eqref{eq:BC}
generalizes the Burkhardt-Cottingham (BC) sum rule~\cite{Burkhardt:1970ti},
which states that  $\int_0^1 dx\, g_2(x) =0$. 
However, we show that jet-mass corrections violate the BC sum rule. The
possibility of a violation of the sum rule due to 
contributions from spin-flip
processes was already mentioned in the original
derivation~\cite{Burkhardt:1970ti}, 
but do not show up in treatments that only
consider free field quark propagators for the struck quark
\cite{Jaffe:1996zw}. Since $h_1$ is driven to 0 by QCD evolution as $Q^2 \to
\infty$, the BC sum rule $\int_0^1 dx\, g_2(x) =0$ is satisfied at least
asymptotically. Although we formulated \eqref{eq:BC} in terms of sum over quark flavors in order to display a clear
connection to the structure function $g_2$, we stress that it is valid also
flavor by flavor, i.e., for each single flavor the only measurable nonzero contribution to the zeroth moment of the
structure function $g_2$ can come from the jet-mass
corrections and transversity.\footnote{This conclusion is true even if the BC sum rule is broken by a $J = 0$ fixed pole with
non-polynomial residue \cite{Jaffe:1996zw}, since this would appear as a
$\delta(x)$ contribution and would not be measurable.}
The fact that $g_2^{\text{quark}}$ disappears in the zeroth moment
of $g_2$ is due to the definition of the $\ast$ functions. The fact that $g_2^{\rm tw3}$
disappears is due to the symmetry properties of $\widehat{g}_T^q$, or
equivalently  is a consequence of the
Lorentz invariance of QCD interactions, that entails $\int_0^1 dx g_1^a(x) =
\int_0^1 g_T^q(x)$.

At finite scales, the only way to preserve the validity of the
Burkhardt-Cottingham sum rule is if
%that is derived at the cross section level (under suitable assumptions on the
%high-energy behavior of te hspin-flip amplitude) rather than from an analysis
%of collinear correlators, as done here, 
%to be strictly valid, we obtain a
%strong constraint on the transversity function, 
\begin{align}
   \int dx\, \frac{1}{x} h_1^q(x) = 0 \ .
\label{eq:ABsumrule}
\end{align}
Interestingly, one can show that this constraint, if valid at any given scale
$Q_0$ is conserved through QCD evolution. However, we think that this
constrain cannot be satisfied in general, since it is broken in perturbative
QCD~\cite{Kundu:2001pk} and models (see, e.g., \cite{}).

E or explicitly as, \eg, in the quark
target model \cite{Kundu:2001pk} -- i.e., if $g_2$ integrates to a finite but
nonzero number -- 
If we assume that the BC sum rule is broken by a {\em finite} amount, we
obtain that $h_1(x)/x$ must be integrable, implying a 
a bound on the small $x$ behavior of the transversity, 
\begin{align}
  h_1^q(x) \propto x^\epsilon \ \ \ \epsilon>0 \ .
\label{eq:ABbound}
\end{align}
This bound will be very useful, \eg, in transversity fits, where the data at
small $x$ is, as yet, very limited, and in general for proper extrapolations
when calculating moments. 

The first moment is the first in which a contribution from the pure twist-3 part of $g_2$ appears:
\begin{align}
  d_1 & = \frac12 \sum_q e_q^2 \bigg( 2 \tilde g_T^q[1] + \hat g_T^q[1]
    + \frac{2M_q-m_q}{\Lambda} h_1^q[1] \bigg)
\end{align}
where $h_1^q[1] = \int_0^1 dx h_1^q(x)$ is the contribution of a quark $q$ to the tensor charge. 
The third moment is also interesting because the pure twist-3 part can be related to quark-gluon-quark correlators, see \cite{Jaffe:1996zw}, and interpreted as as the average color force experienced by the struck quark as it exits the nucleon \cite{Burkardt:2012sd}:
\begin{align}
  d_2 & = \frac12 \sum_q e_q^2 \bigg( 3 \tilde g_T^q[2] + \hat g_T^q[2]
    + \frac{3M_q-m_q}{\Lambda} h_1^q[2] \bigg)
\end{align}
In both cases, the transversity contribution is a background to the extraction of the pure twist-3 piece. Fortunately, it is a quantity that can be extracted from the lattice
\cite{Green:2012ej,Bali:2014nma,Bhattacharya:2015wna,Abdel-Rehim:2015owa,Bhattacharya:2016zcn} or fitted
\cite{Radici:2015mwa,Anselmino:2015sxa,Kang:2015msa}.
Furthermore, the new sum rule \eqref{eq:ABsumrule} and the bound \eqref{eq:ABbound} promise to improve future transversity fits. What is less obvious is how to calculate $M_q-m_q$ from first principles, since this is related to the spectral function of the quark propagator. We will, however, briefly discuss perspectives on how to measure it. Therefore the pure twist-3 part can, in principle, be properly isolated.


\begin{figure}[tbh]
  \centering
  \parbox[b]{8cm}{\centering
  \includegraphics
      [width=7cm,clip=true]
      {epl_emn_h_X}
  }
  \parbox[b]{8cm}{\vskip-0.5cm\centering
  \includegraphics
      [width=7cm,clip=true]
      {epl_emn_h_h_X}
  }
  \caption{\small
	Single hadron {\it (left)} and double hadron {\it (right)}
	production in $e^+ e^-$ collisions at LO with jet and
	fragmentation correlators.
  }
  \label{fig:epl_emn_h}
\end{figure}

A promising avenue to experimentally access jet functions is, however, through inclusive single hadron production, $e^+ e^- \to h X$, and inclusive dihadron 
production from the same hemisphere, $e^+ e^- \to h h X$, see 
Fig.~\ref{fig:epl_emn_h}.
In single-hadron production, the fragmentation functions $D_h$
play the role of PDFs in DIS, and couple to the jet functions
in an analogous way.
In double hadron production the enlarged number of Dirac structures
of the dihadron fragmentation correlator $D_{2h}$ allows one to access
the jet function in novel ways, and in particular to isolate the
contribution from the helicity-flip $J_0$ term.
Studying and classifying all the possibilities offered by single and
double hadron production will open up a rich phenom, which will in turn
be needed to extract pure twist-3 matrix elements from the $g_2$ structure function, and more in general to perform precise jet mass corrections in DIS.

{\bf [---------------- AA:  EDITED down here --------------------]}


It is important to explore in which other process does $\mj$ contribute, as to provide an experimental check of the formalism:
\begin{itemize}
\item inclusive $\Lambda$ production in $e^+ + e^-$
\item same-side dihadrons in $e^+ + e^-$ 
\end{itemize}
\todo{It would be cool to find a process where the $\mj$ contribution is the only one (similar to the BC breaking) ...}   



%%%%%%%%%%%%%%%%%%%%%%%%%%%%%%%%%%%%%%%%%%%%%%%%%%%%%%%%%%%%%%%%%%%%%%%%%
\section{Conclusions}


%%%%%%%%%%%%%%%%%%%%%%%%%%%%%%%%%%%%%%%%
%%%%%%%%% ACKNOWLEDGMENTS %%%%%%%%%%%%%%
%%%%%%%%%%%%%%%%%%%%%%%%%%%%%%%%%%%%%%%%

\begin{acknowledgments}
This work was supported by DOE contract No. DE-AC05-06OR23177,
under which Jefferson Science Associates, LLC operates Jefferson Lab, by the DOE contract DE-SC008791 and 
by the European Research Council (ERC) under the European Union's 
Horizon 2020 research and innovation programme (grant agreement No. 647981,
3DSPIN)
\end{acknowledgments}


\bibliographystyle{myrevtex}
\bibliography{mybiblio}

\end{document}
